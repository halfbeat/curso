\chapter{Presentaci\'{o}n}
\label{presentacion}

%{
%\parindent 0em
%\hangafter -6
%\hangindent 1in
%\lipsum
%}

En los �ltimos tiempos se han realizado avances significativos en la simplificaci�n de los desarrollos Java de aplicaciones empresariales. 
La Junta de Castilla y Le�n sigue anclada en tecnolog�as de desarrollo muy desfasadas con respecto a las actuales tecnolog�as. 
Este curso pretende ayudar a los desarrolladores de aplicaciones en el uso de nuevas tecnolog�as que simplifican mucho los desarrollos software. 

Las aplicaciones que desarrollamos habitualmente se componen de tres grandes bloques diferenciados: La capa de vista, la l�gica de negocio y la capa acceso a datos. 
Este curso se orienta a la capa de vista y a la de acceso a datos. JSF da soluci�n a la capa de vista mientras que JPA lo hace en la capa de acceso a datos. 

El enfoque de este curso es eminentemente pr�ctico. El curso se divide en dos grandes bloques, uno referido a JPA y el otro a JSF. 

En el primer bloque aprenderemos como modelar entidades persistentes en base de datos, como crearlax, como acceder a ellas, como modificarlas y como eliminarlas. 

En el segundo aprenderemos a crear interfaces de usuario basadas en el est�ndar JSF, a utilizar platillas reutilizables, a utilizar controles mejorados y a darles soporte Ajax.


\section*{Requisitos del curso} 

Este curso est� dirigido a desarrolladores de aplicaciones J2EE y tiene un enfoque pr�ctico. Por ello se asume que los alumnos tienen conocimientos de programaci�n Java y JSP. 

También se presupone que los alumnos tienen conocimientos de modelado de datos relacionales y lenguaje SQL.

El cambio tecnol�gico nos impone el uso de la versi�n 5 de Java. El uso de esta versi�n no supone apenas cambios en el desarrollo aunque utilizaremos la capacidad 
de anotaciones de esta versi�n en la parte de JPA. Al final del libro hay un apéndice con un resumen de las caracter�sticas de Java 5 que vamos a utilizar y 
ejemplos de cada una de ellas.

Es impensable en la actualidad el trabajar sin la ayuda de algún IDE de desarrollo que nos facilite la labor. Hist�ricamente en la Junta de Castilla y 
Le�n se ha utilizado JDeveloper en cualquiera de sus versiones. En este curso no utilizaremos dicho IDE sino que utilizaremos Eclipse. Este IDE adem�s de 
soportar plenamente las tecnolog�as que vamos a ver en este curso, siendo adem�s mucho m�s vers�til.


