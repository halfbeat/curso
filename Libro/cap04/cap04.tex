\chapter{Teor�a de vista: JSF 2}

\section{El ciclo de vida de una request}
Una aplicaci\'on \emph{JSF} no es m\'as que una aplicaci\'on Web que
utiliza un \emph{Servlet }denominado \emph{FacesServlet }el cual
recibe las peticiones desde el browser del usuario a la aplicaci\'on.
Por tanto la configuraci\'on de la aplicaci\'on se realiza en el
fichero \emph{web.xml} de la forma habitual.

Una vez que una \emph{request} llega al \emph{servlet }esta pasa por
una serie de fases cuya ejecuci\'on depende de la p\'agina se est\'a
presentando por primera vez o estamos procesando un formulario.

El siguiente diagrama muestra el ciclo de vida de una \emph{request}
\emph{JSF}:


{\centering 
\includegraphics[width=11.795cm,height=6.535cm]{cap04/jsf-img1.png} \par}

El ciclo de vida gestiona dos tipos de \emph{request}: las
\emph{initial requests} y \emph{postbacks}. Cuando un usuario
realiza una \emph{initial request} est\'a \ pidiendo la p\'agina por
primera vez. Cuando el usuario ejecuta un \emph{postback}, se env\'ia
el formulario contenido en la p\'agina que ha sido cargada previamente
en el navegador como resultado de la ejecuci\'on de una \emph{initial
request}. Cuando el ciclo de vida procesa una \emph{initial}
\emph{request}, solamente se ejecutan las fases \emph{restore}
\emph{view} y \emph{render} ya que no hay entrada de usuario o
acciones que procesar. Cuando el ciclo de vida procesa una 
\emph{postback} \emph{request} se ejecutan las siguientes fases: 

{\centering \includegraphics[width=13.039cm,height=9.342cm]{cap04/jsf-img2.png} \par}

\subsection*{Restore View Phase }
Cuando se realiza una solicitud de una p\'agina \emph{JavaServer}
\emph{Faces}, por ejemplo, cuando se pulsa un enlace o un bot\'on, la
implementaci\'on \emph{JavaServer} \emph{Faces} comienza la fase de
restauraci\'on de la vista.

Durante esta fase, la implementaci\'on \emph{JavaServer}
\emph{Faces} construye la vista de la p\'agina, los enlaces a los
eventos, los validadores de los componentes de la vista, y guarda la
vista en la instancia \emph{FacesContext}. La instancia
\emph{FacesContext} contiene toda la informaci\'on necesaria para
procesar una \'unica solicitud. Todas las etiquetas de los componentes
de la aplicaci\'on, los controladores de eventos, conversores y
validadores tienen acceso a la instancia \emph{FacesContext}. 
 
Si la solicitud de la p\'agina es una \emph{initial request}, la
implementaci\'on \emph{JavaServer} \emph{Faces} crea una vista
vac\'ia durante esta etapa y avanza directamente el ciclo de vida hasta
la etapa de \emph{render response}.

Si la solicitud de la p\'agina es una \emph{postback}
\emph{request}, durante esta fase, la implementaci\'on
\emph{JavaServer} \emph{Faces} restaura la vista mediante el uso de
la informaci\'on de estado guardada en el cliente o el servidor.

\subsection*{Apply Request Values Phase }
Despu\'es de restaurar \ el \'arbol de componentes, cada componente del
\'arbol extrae su nuevo valor de los par\'ametros de la
\emph{request} usando el m\'etodo \emph{decode}. El valor se
almacena localmente en el componente. Si la conversi\'on del valor
falla, se genera un mensaje de error asociado al componente y se encola
en \emph{FacesContext}. Este mensaje se mostrar\'a durante la fase de
\emph{response}, junto con los errores de validaci\'on resultantes de
la fase de validaci\'on. 

Si hay eventos han encolados durante esta fase, la implementaci\'on
\emph{JavaServer} \emph{Faces} env\'ia los eventos a los
componentes interesados. 

Si algunos de los componentes de la p\'agina tiene el atributo
\emph{inmediate} definido a \emph{true}, entonces la validaci\'on,
la conversi\'on y los eventos asociados a estos componentes ser\'an
procesados durante esta fase. 

Al final de esta fase, los componentes se cargan con sus nuevos valores,
y los mensajes y los acontecimientos han sido encolados.

\subsection*{Process Validations Phase }
Durante esta fase, la implementaci\'on \emph{JSF} procesa todas las
validaciones registradas en el \'arbol de componentes, examina los
atributos de los componentes que especifican las reglas de validaci\'on
y compara esas reglas con los valores almacenados en el componente.

Si el valor almacenado es inv\'alido, la implementaci\'on inserta un
mensaje de error en la instancia \emph{FacesContext} y el ciclo de
vida contin\'ua directamente hacia la fase de respuesta para que la
p\'agina sea generada de nuevo con los mensajes de error. Si hubiera
errores de conversi\'on generados en la fase anterior esos mensajes de
error se mostrar\'ian tambi\'en en esta fase. Si se produjesen eventos
en esta fase, la implementaci\'on los dirigir\'a hacia los componentes
interesados. 

\subsection*{Update Model Values Phase }
Una vez que la implementaci\'on determina que los datos son correctos,
recorrer\'a el \'arbol de componentes y establecer\'a (m\'etodo
\emph{set}) el valor de la propiedad correspondiente \ en el objeto
del servidor (\emph{bean}). Solo se actualizar\'an \ los atributos
apuntados por componentes de entrada. Si el valor no puede ser
convertido al tipo especificado por el atributo del \emph{bean}, el
ciclo de vida avanza a la fase de respuesta por lo que los errores
ser\'a mostrados con la respuesta. Lo mismo ocurre con los errores de
validaci\'on. Si se produjesen eventos en esta fase, la
implementaci\'on los dirigir\'a hacia los componentes interesados.

\subsection*{Invoke Application Phase }
Durante esta fase, la implementaci\'on \emph{JSF} maneja los eventos
de nivel de aplicaci\'on, como el env\'io de un formulario o el enlace
a otra p\'agina. 

En esta fase se comprueban las reglas de navegaci\'on y la
implementaci\'on \emph{JSF} transferir\'a el control a la fase de
respuesta estableciendo la vista a la p\'agina actual o la nueva
obtenida por navegaci\'on. 

\subsection*{Render Response Phase }
En esta fase la implementaci\'on \emph{JSF} delega la composici\'on de
la p\'agina al contenedor JSP en el caso de que la aplicaci\'on utilice
p\'aginas JSP \ (lo m\'as habitual). Si se trata de una
\emph{initial} \emph{request} los componentes que representan la
p\'agina \ ser\'an a\~nadidos al \'arbol de componentes sobre el que el
contenedor JSP ejecuta la p\'agina. Si no es una \emph{initial}
\emph{request}, los componentes ya se encuentran en el \'arbol por lo
que no es necesario a\~nadirlos de nuevo. En cualquier caso los
componentes se muestran a s\'i mismos durante el recorrido por el
\'arbol que realizara el contenedor JSP leyendo los tags de la
p\'agina.
 
Si la \emph{request} es de tipo \emph{postback} y se detectan
errores resultantes de la ejecuci\'on de cualquiera de las fases
anteriores, la p\'agina original es enviada al browser. Si en dicha
p\'agina apareciesen componentes
\emph{{\textless}h:message/{\textgreater}} o
\emph{{\textless}h:messages{\textgreater},} los mensajes de error se
mostrar\'ian en la p\'agina.

Antes de que la vista sea generada, el estado de la respuesta se
almacena para que las siguientes \emph{request}s lo tengan disponible
en la fase de respuesta.

\section{Configuraci�n de aplicaciones Web con tecnolog�a JSF 2}

Para configurar un proyecto en \textit{JSF} necesitamos una serie de
cosas en nuestro proyecto J2EE:

\begin{itemize}
\item Implementaci\'on \textit{JSF}
\item Fichero \textit{web.xml} configurado para la implementaci\'on
\textit{JSF} que vamos a utilizar
\item Fichero \textit{faces-config.xml}
\end{itemize}
\subsection[Implementaci\'on JSF]{Implementaci\'on \textit{JSF}}
Existen varias implementaciones \textit{JSF} de la especificaci\'on. Las
versiones m\'as utilizadas son: 

\textbf{sun-ri}\ \ \ \ Implementaci\'on de Sun

\textbf{myfaces}\ \ Implementaci\'on de myfaces

Para las pr\'acticas de este curso utilizaremos la implementaci\'on de
Sun.

\subsection[Fichero web.xml]{Fichero \textit{web.xml}}
El fichero \textit{web.xml} debe de ser modificado para utilizar el
\textit{servlet }que ofrece la implementaci\'on de \textit{JSF}. Las
entradas m\'inimas son las siguientes


\bigskip
\begin{framed}
\lstset{language=XML}
\begin{lstlisting}
<context-param>
	<param-name>javax.faces.DEFAULT_SUFFIX</param-name>
	<param-value>xhtml</param-value>
  </context-param>
<servlet>
	<servlet-name>Faces Servlet</servlet-name>
	<servlet-class>javax.faces.webapp.FacesServlet</servlet-class>
	<load-on-startup>1</load-on-startup>
</servlet>
<servlet-mapping>
	<servlet-name>Faces Servlet</servlet-name>
	<url-pattern>*.faces</url-pattern>
</servlet-mapping>
\end{lstlisting}
\end{framed} 

\bigskip

\begin{framed}
Con este ejemplo decimos a nuestro servidor que todas las peticiones
cuya URL acabe en .\textit{faces} se procesen por el Servet
FacesServlet.

Adem\'as decimos que la extensi\'on de nuestras p\'aginas web ser\'a
xhtml. Podr\'iamos no utilizar xhtml simplemente no declarando este
context-param, y utilizar jsp, pero todo este curso se basa en la
utilizaci\'on de p\'aginas XHTML por las ventajas que aporta.
\end{framed}


\bigskip

{\itshape
Utilizar xhtml implica que nuestras p\'aginas web estar\'an escritas en
XHTML. A partir de ahora ser\'an de hecho ficheros xml que ser\'an
procesados de forma estricta por el servidor. Una p\'agina xhtml tiene
el siguiente aspecto:}


\bigskip

\lstset{language=XML}
\begin{lstlisting}
<?xml version="1.0" encoding="ISO-8859-1" ?>
<!DOCTYPE html PUBLIC "-//W3C//DTD XHTML 1.0 Transitional//EN" 
    "http://www.w3.org/TR/xhtml1/DTD/xhtml1-transitional.dtd">
<html xmlns="http://www.w3.org/1999/xhtml">
	<body>
	</body>
</html>
\end{lstlisting} 

{\itshape
{\textless}?\textcolor[rgb]{0.24705882,0.49803922,0.49803922}{xml}
\textcolor[rgb]{0.49803922,0.0,0.49803922}{version}\textcolor{black}{=}\textcolor[rgb]{0.16470589,0.0,1.0}{1.0}
\textcolor[rgb]{0.49803922,0.0,0.49803922}{encoding}\textcolor{black}{=}\textcolor[rgb]{0.16470589,0.0,1.0}{ISO-8859-1}
?{\textgreater}}

{\itshape
\textcolor{black}{{\textless}!DOCTYPE html PUBLIC
-//W3C//DTD XHTML 1.0 Transitional//EN }}

{\itshape
\textcolor{black}{\ \ \ \ http://www.w3.org/TR/xhtml1/DTD/xhtml1-transitional.dtd{\textgreater}}}

{\itshape
{\textless}\textcolor[rgb]{0.24705882,0.49803922,0.49803922}{html}
\textcolor[rgb]{0.49803922,0.0,0.49803922}{xmlns}\textcolor{black}{=}\textcolor[rgb]{0.16470589,0.0,1.0}{http://www.w3.org/1999/xhtml{\textquotedblright}{\textgreater}}}

{\itshape
\textcolor{black}{\ \ }{\textless}\textcolor[rgb]{0.24705882,0.49803922,0.49803922}{body}{\textgreater}}

{\itshape
\textcolor{black}{\ \ }{\textless}/\textcolor[rgb]{0.24705882,0.49803922,0.49803922}{body}{\textgreater}}

{\itshape
{\textless}/\textcolor[rgb]{0.24705882,0.49803922,0.49803922}{html}{\textgreater}}

\bigskip

{\itshape
Utilizaremos los namespaces del fichero xhtml (xml) para acceder a los
diferentes tags que necesitar\'an las p\'aginas (tags de JSF,
richfaces, ajax, facelets, ..). Todo esto se ver\'a m\'as adelante.}


\bigskip

Vamos a ver una pr\'actica para familiarizarnos con un proyecto JSF
b\'asico.


\bigskip

{\raggedleft\bfseries\itshape
Pr\'actica 1: Hola Mundo
\par}

{\itshape
Crear un proyecto con la configuraci\'on m\'inima y una pantalla de
inicio que presente el texto {\textquotedblleft}Hola
mundo{\textquotedblright}.}

{\raggedleft\bfseries\itshape
Soluci\'on
\par}

{\itshape
La soluci\'on implica crear la siguiente estructura de proyecto:}


\bigskip

{\centering 
\includegraphics[width=4.784cm,height=4.443cm]{cap04/jsf-imga1.png} \par}


\bigskip

{\itshape
\textbf{Fichero web.xml:}}

{\itshape
{\textless}?\textcolor[rgb]{0.24705882,0.49803922,0.49803922}{xml}
\textcolor[rgb]{0.49803922,0.0,0.49803922}{version}\textcolor{black}{=}\textcolor[rgb]{0.16470589,0.0,1.0}{1.0}
\textcolor[rgb]{0.49803922,0.0,0.49803922}{encoding}\textcolor{black}{=}\textcolor[rgb]{0.16470589,0.0,1.0}{UTF-8}?{\textgreater}}

{\itshape
{\textless}!\textcolor[rgb]{0.24705882,0.49803922,0.49803922}{DOCTYPE}
web-app \textcolor[rgb]{0.5019608,0.5019608,0.5019608}{PUBLIC}
-//Sun Microsystems, Inc.//DTD Web Application
2.3//EN
\textcolor[rgb]{0.24705882,0.49803922,0.37254903}{http://java.sun.com/dtd/web-app\_2\_3.dtd}{\textgreater}}

{\itshape
{\textless}\textcolor[rgb]{0.24705882,0.49803922,0.49803922}{web-app}
\textcolor[rgb]{0.49803922,0.0,0.49803922}{id}\textcolor{black}{=}\textcolor[rgb]{0.16470589,0.0,1.0}{WebApp\_ID}{\textgreater}}

{\itshape
\textcolor{black}{\ \ }{\textless}\textcolor[rgb]{0.24705882,0.49803922,0.49803922}{context-param}{\textgreater}}

{\itshape
\textcolor{black}{\ \ \ \ }{\textless}\textcolor[rgb]{0.24705882,0.49803922,0.49803922}{param-name}{\textgreater}\textcolor{black}{javax.Faces.DEFAULT\_SUFFIX}{\textless}/\textcolor[rgb]{0.24705882,0.49803922,0.49803922}{param-name}{\textgreater}}

{\itshape
\textcolor{black}{\ \ \ \ }{\textless}\textcolor[rgb]{0.24705882,0.49803922,0.49803922}{param-value}{\textgreater}\textcolor{black}{.jspx}{\textless}/\textcolor[rgb]{0.24705882,0.49803922,0.49803922}{param-value}{\textgreater}}

{\itshape
\textcolor{black}{\ \ }{\textless}/\textcolor[rgb]{0.24705882,0.49803922,0.49803922}{context-param}{\textgreater}}

{\itshape
\textcolor{black}{\ \ }{\textless}\textcolor[rgb]{0.24705882,0.49803922,0.49803922}{servlet}{\textgreater}}

{\itshape
\textcolor{black}{\ \ \ \ }{\textless}\textcolor[rgb]{0.24705882,0.49803922,0.49803922}{servlet-name}{\textgreater}\textcolor{black}{Faces
}\textcolor{black}{Servlet}{\textless}/\textcolor[rgb]{0.24705882,0.49803922,0.49803922}{servlet-name}{\textgreater}}

{\itshape
\textcolor{black}{\ \ \ \ }{\textless}\textcolor[rgb]{0.24705882,0.49803922,0.49803922}{servlet-class}{\textgreater}\textcolor{black}{javax.Faces.webapp.FacesServlet}{\textless}/\textcolor[rgb]{0.24705882,0.49803922,0.49803922}{servlet-class}{\textgreater}}

{\itshape
\textcolor{black}{\ \ \ \ }{\textless}\textcolor[rgb]{0.24705882,0.49803922,0.49803922}{load-on-startup}{\textgreater}\textcolor{black}{1}{\textless}/\textcolor[rgb]{0.24705882,0.49803922,0.49803922}{load-on-startup}{\textgreater}}

{\itshape
\textcolor{black}{\ \ }{\textless}/\textcolor[rgb]{0.24705882,0.49803922,0.49803922}{servlet}{\textgreater}}

{\itshape
\textcolor{black}{\ \ }{\textless}\textcolor[rgb]{0.24705882,0.49803922,0.49803922}{servlet-mapping}{\textgreater}}

{\itshape
\textcolor{black}{\ \ \ \ }{\textless}\textcolor[rgb]{0.24705882,0.49803922,0.49803922}{servlet-name}{\textgreater}\textcolor{black}{Faces
}\textcolor{black}{Servlet}{\textless}/\textcolor[rgb]{0.24705882,0.49803922,0.49803922}{servlet-name}{\textgreater}}

{\itshape
\textcolor{black}{\ \ \ \ }{\textless}\textcolor[rgb]{0.24705882,0.49803922,0.49803922}{url-pattern}{\textgreater}\textcolor{black}{*.Faces}{\textless}/\textcolor[rgb]{0.24705882,0.49803922,0.49803922}{url-pattern}{\textgreater}}

{\itshape
\textcolor{black}{\ \ }{\textless}/\textcolor[rgb]{0.24705882,0.49803922,0.49803922}{servlet-mapping}{\textgreater}}

{\itshape
{\textless}/\textcolor[rgb]{0.24705882,0.49803922,0.49803922}{web-app}{\textgreater}}


\bigskip

{\itshape
\textbf{Fichero faces-config.xml:}}

{\itshape
{\textless}?\textcolor[rgb]{0.24705882,0.49803922,0.49803922}{xml}
\textcolor[rgb]{0.49803922,0.0,0.49803922}{version}\textcolor{black}{=}\textcolor[rgb]{0.16470589,0.0,1.0}{1.0}
\textcolor[rgb]{0.49803922,0.0,0.49803922}{encoding}\textcolor{black}{=}\textcolor[rgb]{0.16470589,0.0,1.0}{UTF-8}?{\textgreater}}

{\itshape
{\textless}!\textcolor[rgb]{0.24705882,0.49803922,0.49803922}{DOCTYPE}
faces-config \textcolor[rgb]{0.5019608,0.5019608,0.5019608}{PUBLIC}}

{\itshape
\ \ \ \ -//Sun Microsystems, Inc.//DTD JavaServer Faces
Config 1.1//EN}

{\itshape
\ \ \ \ \textcolor[rgb]{0.24705882,0.49803922,0.37254903}{http://java.sun.com/dtd/web-facesconfig\_1\_1.dtd}{\textgreater}}

{\itshape
{\textless}\textcolor[rgb]{0.24705882,0.49803922,0.49803922}{faces-config}{\textgreater}}

{\itshape
\textcolor{black}{\ \ }{\textless}\textcolor[rgb]{0.24705882,0.49803922,0.49803922}{application}{\textgreater}}

{\itshape
\textcolor{black}{\ \ \ \ }{\textless}\textcolor[rgb]{0.24705882,0.49803922,0.49803922}{view-handler}{\textgreater}\textcolor{black}{com.sun.facelets.FaceletViewHandler}{\textless}/\textcolor[rgb]{0.24705882,0.49803922,0.49803922}{view-handler}{\textgreater}}

{\itshape
\textcolor{black}{\ \ }{\textless}/\textcolor[rgb]{0.24705882,0.49803922,0.49803922}{application}{\textgreater}}

{\itshape
{\textless}/\textcolor[rgb]{0.24705882,0.49803922,0.49803922}{faces-config}{\textgreater}}


\bigskip

{\itshape
\textbf{Fichero index.jspx:}}

{\itshape
{\textless}?\textcolor[rgb]{0.24705882,0.49803922,0.49803922}{xml}
\textcolor[rgb]{0.49803922,0.0,0.49803922}{version}\textcolor{black}{=}\textcolor[rgb]{0.16470589,0.0,1.0}{1.0}
\textcolor[rgb]{0.49803922,0.0,0.49803922}{encoding}\textcolor{black}{=}\textcolor[rgb]{0.16470589,0.0,1.0}{ISO-8859-1}
?{\textgreater}}

{\itshape
{\textless}!\textcolor[rgb]{0.24705882,0.49803922,0.49803922}{DOCTYPE}
html \textcolor[rgb]{0.5019608,0.5019608,0.5019608}{PUBLIC}
-//W3C//DTD XHTML 1.0 Transitional//EN }

{\itshape
\ \ \ \ \textcolor[rgb]{0.24705882,0.49803922,0.37254903}{http://www.w3.org/TR/xhtml1/DTD/xhtml1-transitional.dtd}{\textgreater}}

{\itshape
{\textless}\textcolor[rgb]{0.24705882,0.49803922,0.49803922}{html}
\textcolor[rgb]{0.49803922,0.0,0.49803922}{xmlns}\textcolor{black}{=}\textcolor[rgb]{0.16470589,0.0,1.0}{http://www.w3.org/1999/xhtml}}

{\itshape
\ \ \textcolor[rgb]{0.49803922,0.0,0.49803922}{xmlns:h}\textcolor{black}{=}\textcolor[rgb]{0.16470589,0.0,1.0}{http://java.sun.com/JSF/html}}

{\itshape
\ \ \textcolor[rgb]{0.49803922,0.0,0.49803922}{xmlns:f}\textcolor{black}{=}\textcolor[rgb]{0.16470589,0.0,1.0}{http://java.sun.com/JSF/core}{\textgreater}}

{\itshape
\textcolor{black}{\ \ }{\textless}\textcolor[rgb]{0.24705882,0.49803922,0.49803922}{body}{\textgreater}}

{\itshape
\textcolor{black}{\ \ \ \ }{\textless}\textcolor[rgb]{0.24705882,0.49803922,0.49803922}{h1}{\textgreater}\textcolor{black}{Hola}\textcolor{black}{
}\textcolor{black}{mundo}{\textless}/\textcolor[rgb]{0.24705882,0.49803922,0.49803922}{h1}{\textgreater}}

{\itshape
\textcolor{black}{\ \ }{\textless}/\textcolor[rgb]{0.24705882,0.49803922,0.49803922}{body}{\textgreater}}

{\itshape
{\textless}/\textcolor[rgb]{0.24705882,0.49803922,0.49803922}{html}{\textgreater}}


\bigskip

{\itshape
Como esta es la primera pr\'actica vamos a ver paso a paso como crear el
proyecto en eclipse:}


\bigskip

{\raggedleft\bfseries\itshape
Paso 1: Crear el proyecto
\par}

{\itshape
Los ficheros del curso estar\'an ubicados en la carpeta
C:{\textbackslash}CURSO}

{\itshape
En dicha carpeta existir\'a otra carpeta de nombre
C:{\textbackslash}CURSO{\textbackslash}PRACTICAS{\textbackslash}JSF}

{\itshape
Y otra con el nombre
C:{\textbackslash}CURSO{\textbackslash}PRACTICAS{\textbackslash}eclipse}

{\itshape
Abrimos el eclipse
(C:{\textbackslash}CURSO{\textbackslash}SOFTWARE{\textbackslash}eclipse)
y obtendremos una ventana similar a esta:}


\bigskip

{\centering 
\includegraphics[width=11.128cm,height=6.645cm]{cap04/jsf-imga2.png} \par}

{\itshape
Seleccionamos File [F0E0?] \ New [F0E0?]Dynamic Web Project. Nos
presentar\'a \ una ventana que dejaremos de la siguiente forma:}


\bigskip

{\centering 
\includegraphics[width=7.913cm,height=9.356cm]{cap04/jsf-imga3.png} \par}


\bigskip

{\itshape
Observar que se ha cambiado la ubicaci\'on del proyecto a
C:{\textbackslash}CURSO{\textbackslash}PRACTICAS{\textbackslash}JSF }

{\itshape
Pulsamos siguiente y obtenemos:}


\bigskip

{\centering 
\includegraphics[width=7.992cm,height=9.451cm]{cap04/jsf-imga4.png} \par}


\bigskip

{\itshape
Pulsamos siguiente y lo dejamos como en la siguiente imagen:}


\bigskip

{\centering 
\includegraphics[width=9.999cm,height=5.459cm]{cap04/jsf-imga5.png} \par}


\bigskip

{\itshape
Pulsamos finalizar y terminamos con la creaci\'on del proyecto que
quedar\'a:}


\bigskip

{\centering  \includegraphics[width=12.3cm,height=6.267cm]{cap04/jsf-imga6.png}
\par}


\bigskip

{\itshape
Necesitamos a\~nadir las librer\'ias al proyecto. Tenemos dos opciones:
}

\begin{itemize}
\item {\itshape
configurar un library path }
\item {\itshape
copiar los jars \ al directorio WEB-INF/lib.}
\end{itemize}
{\itshape
\ Vamos a utilizar la segunda opci\'on por su simplicidad. Copiamos el
contenido (no la carpeta, solo los jars) de la carpeta
C:{\textbackslash}CURSO{\textbackslash}PRACTICAS{\textbackslash}libs{\textbackslash}JSF-libs
a
C:{\textbackslash}CURSO{\textbackslash}PRACTICAS{\textbackslash}JSF{\textbackslash}PracticaJSF1{\textbackslash}web{\textbackslash}WEB-INF{\textbackslash}lib
y en el proyecto seleccionamos dicha carpeta y pulsamos F5 para que
actualice el contenido. Veremos lo siguiente:}


\bigskip

{\centering 
\includegraphics[width=9.999cm,height=9.109cm]{cap04/jsf-imga7.png} \par}


\bigskip

{\itshape
Para este proyecto no necesitamos todas las librer\'ias pero las
utilizaremos en los siguientes.}

{\raggedleft\bfseries\itshape
Paso 2: Realizaci\'on de la pr\'actica.
\par}

{\itshape
Modificamos el fichero web.xml como se ha explicado antes.}

{\itshape
Vamos a crear el fichero faces-config.xml. Seleccionamos la carpeta
WEB-INF en el \'arbol seleccionamos File[F0E0?]New[F0E0?]File}

{\itshape
Dejamos la ventana de esta forma:}


\bigskip

{\centering 
\includegraphics[width=7.689cm,height=8.841cm]{cap04/jsf-imga8.png} \par}


\bigskip

{\itshape
Y pulsamos Finalizar. Hacemos doble click sobre el fichero reci\'en
creado y lo dejamos como se ha explicado anteriormente.}

{\itshape
Ahora solo nos falta crear el fichero index.html. Realizamos el mismo
proceso pero sobre la carpeta web. El proyecto deber\'ia tener el
siguiente aspecto:}


\bigskip

{\centering 
\includegraphics[width=9.537cm,height=7.811cm]{cap04/jsf-imga9.png} \par}

{\raggedleft\bfseries\itshape
Paso 3: Ejecuci\'on del proyecto
\par}

{\itshape
Para ello necesitamos configurar un servidor. Esto solo se hace la
primera vez.}

{\itshape
Vamos a utilizar un servidor Tomcat para ejecutar nuestras pr\'acticas.
Hacemos lo siguiente:}

{\itshape
Seleccionamos la opci\'on New[F0E0?]Server con el bot\'on derecho del
rat\'on en la pesta\~na Servers}

{\itshape
En la nueva ventana seleccionamos Tomcat 6:}


\bigskip

{\centering 
\includegraphics[width=7.992cm,height=8.407cm]{cap04/jsf-imga10.png} \par}


\bigskip

{\itshape
Pulsamos siguiente y seleccionamos la ruta al Tomcat
(C:{\textbackslash}CURSO{\textbackslash}SOFTWARE{\textbackslash}apache-tomcat-6.0.29.}


\bigskip

{\centering 
\includegraphics[width=8.805cm,height=9.239cm]{cap04/jsf-imga11.png} \par}


\bigskip

{\itshape
Pulsamos siguiente y a\~nadimos nuestro proyecto al servidor:}


\bigskip

{\centering 
\includegraphics[width=9.999cm,height=10.516cm]{cap04/jsf-imga12.png} \par}


\bigskip

{\itshape
Pulsamos finalizar. En la pesta\~na Servers aparecer\'a el nuevo
servidor con nuestro proyecto. Pulsamos con el bot\'on derecho del
rat\'on y seleccionamos Run. Accediendo con un navegador obtendremos lo
siguiente:}


\bigskip

{\centering\itshape
 \includegraphics[width=9.999cm,height=6.616cm]{cap04/jsf-imga13.png} \textup{
}
\par}


\bigskip


\bigskip

\section{Beans gestionados}

\section{P�ginas web en JSF 2}

\subsection{Etiquetas de JSF 2}

\subsection{Lenguaje de Expresiones (EL)}

\section{Navegaci�n}

\section{Conversi�n}
 
\section{Validaci�n}

\section{Gesti�n de errores}

\section{Plantillas en JSF. Facelets}

\section{Utilizaci�n de librer�as JSF de terceros: Primefaces}

 
