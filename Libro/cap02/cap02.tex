\chapter{Teor�a de persistencia: JPA}

\section{Breve historia de la persistencia en Java} 

Desde los comienzos de la programaci�n en Java han aparecido tecnolog�as que permitan el acceso y manipulaci�n (persistencia) de datos ubicados en bases de datos relacionales. 
Un breve repaso de las tecnolog�as de persistencia ser�a el siguiente:
 
\subsection*{JDBC}
La especificaci�n JDBC (Java Database Conectivity) permiti� la estandarizaci�n del acceso a las bases de datos. Siempre que hubiese un driver compatible JDBC para una base de datos,
la especificaci�n nos permite el acceso a la misma desde aplicaciones Java. El problema es que aunque JDBC es un est�ndar, SQL no lo es. 
El c�digo SQL cambia para bases de datos distintas (p.e. MySQL y Oracle tienen sintaxis JAVA ligeramente distintas).
\subsection*{EJBs}
Se introdujo en la primera versi�n de J2EE (Java 2 Enterprise Edition) como nueva soluci�n al problema de la persistencia en forma de Entity Bean. 
Delegaban la persistencia al contenedor aunque con carencias en cuanto a portabilidad (configuraci�n XML ad hoc en los despliegues para proveedores espec�ficos), 
coste de red elevado por el acceso RMI a los beans, mapeo insuficiente de las relaciones entre Entity Beans (foreign keys), etc,\ldots
\subsection*{JDO}
Se trat� de un esfuerzo independiente por dar soluci�n al problema de la persistencia. 
Inicialmente requer�a bases de datos orientadas a objetos aunque posteriormente se ampli� a las bases de datos relacionales. 
Requiere de un proceso de \emph{enhancement} del byte code generado por el compilador java que a�ade datos para la gesti�n de la persistencia  en un proceso posterior a la compilaci�n. 
Tambi�n define un lenguaje de consulta orientado a objetos. JDO alcanz� el status de extensi�n JDK aunque nunca el status de est�ndar Java.
\subsection*{JPA}
JPA es el est�ndar de persistencia desarrollado para la plataforma J2EE mediante el est�ndar EJB3. A diferencia de JDO, JPA es un est�ndar Java (JPA 1.0 JSR220 y JPA 2.0 JSR317) 
existiendo una serie de implantaciones tanto comerciales como libres de dicho est�ndar.
JPA permite mapear los objetos y relaciones entre los mismos a tablas relacionales, permitiendo utilizar POJOs (Plain Old Java Objects) 
para mantener las ventajas de la orientaci�n a objetos en el acceso a la base de datos.
\subsection*{JPA2}

\section{Definiciones}
\subsection*{ManagedEntity}
Entidad gestionada por un EntityManager. 
Se trata de todas las clases que se han declarado como entidades.
\subsection*{EntityManagerFactory}
Objeto que se utiliza para interactuar con la base de datos definida en una unidad 
de persistencia. Normalmente se puede identificar con un DataSource. 
Es la encargada de crear los EntityManagers.
\subsection*{EntityManager}
Gestor de entidades persistentes. Recupera, actualiza y mantiene sincronizados 
los ManagedEntities con la base de datos. 
\subsection*{PersistenceUnit} 
Elemento de agrupaci�n l�gica de entidades persistentes que deben de ser tratadas 
de forma igual. Incluye:
\begin{itemize}
\item
Un EntityManagerFactory y todos sus EntityManagers junto con su informaci�n de configuraci�n.
\item
El conjunto de clases gestionadas incluidas en la unidad de persistencia gestionadas por los EntityManagers del EntityManagerFactory.
\item
Metadatos de mapeo que especifica como se mapean las clases en la base de datos.
\end{itemize}

\section{Configuraci�n de la persistencia}
La configuraci�n de la persistencia se realiza mediante el fichero \emph{persistence.xml}.
En el fichero de configuraci�n se especifican el motor de persistencia que se va a utilizar,
 las entidades que se consideran persistentes, as� como diferentes opciones de configuraci�n
 (logs, weaving, \dots).

\section{Modelado de entidades y sus relaciones}

\subsection{XXX}

\section{Consultas}

\section{Consultas avanzadas. Criteria queries.}
