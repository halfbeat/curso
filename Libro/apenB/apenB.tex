\chapter[Ap�ndice B]{Expression Language [EL]}

\section{Introduction}

\textit{JavaServer} \textit{Faces} provides an expression language
(\textit{JSF} EL) that is used in web application pages to access the
JavaBeans components in the page \textit{bean} and in other beans
associated with the web application, such as the session \textit{bean}
and the application \textit{bean}. 

\section{\textit{JavaServer Faces} EL Expression Syntax}

\textit{JSF} EL can be used to bind JavaBeans to component properties to
simplify how the components access data from various sources.
\textit{JSF} EL expressions use the syntax
\emph{\#\{expr\};} 

The syntax of a value binding expression is identical to the syntax of
an expression language expression defined in the \textit{JavaServer}
Pages Specification (version 2.0), sections 2.3 through 2.9, with the
following exceptions:

\begin{itemize}
\item The expression delimiters for a value binding expression are
\emph{\#\{} and \emph{\}}
instead\newline
of \emph{\$\{} and \emph{\}}. 
\item Value binding expressions do not support JSP expression language
functions. 
\end{itemize}
In addition to the differences in delimiters, the two expression types
have the following semantic differences:

\begin{itemize}
\item During \textit{render}ing, value binding expressions are evaluated
by the \textit{JavaServer} \textit{Faces} implementation (via calls to
the \emph{getValue} method) rather than by the
compiled code for a page. 
\item Value binding expressions can be evaluated programmatically, even
when a page is not present. 
\item Value binding expression evaluation leverages the facilities of
the configured \emph{VariableResolver} and
\emph{PropertyResolver} objects available through
the \emph{Application} object for the current web
application, for which applications can provide plug-in replacement
classes that provide additional capabilities. 
\item If a value binding expression is used for the value property of an
\emph{EditableValueHolder} component (any input
field component), the expression is used to modify the referenced value
rather than to retrieve it during the Update Model Values phase of the
\textit{request} processing lifecycle. \newline
\end{itemize}
Examples of valid value binding expressions include: 

\ \ \ \#\{Page1.name\}

\ \ \ \#\{Foo.bar\}

\ \ \ \#\{Foo[bar]\}

\ \ \ \#\{Foo[bar]\}

\ \ \ \#\{Foo[3]\}

\ \ \ \#\{Foo[3].bar\}

\ \ \ \#\{Foo.bar[3]\}

\ \ \ \#\{Customer.status == VIP\}

\ \ \ \#\{(Page1.City.farenheitTemp - 32) * 5 / 9\}

\ \ \ Reporting Period: \#\{Report.fromDate\} to \#\{Report.toDate\}

For value binding expressions where the
\emph{setValue} method is going to be called (for
example, for \emph{text} property bindings for input
fields during Update Model Values), the syntax of a value binding
expression is limited to one of the following forms, where
\emph{expr-a} is a general expression that evaluates
to some object, and \emph{value-b} is an identifier:

\ \ \ \#\{expr-a.value-b\}

\ \ \ \#\{expr-a[value-b]]

\ \ \ \#\{value-b\}

\section{\bfseries Get Value Semantics}

When the \emph{getValue} method of a
\emph{ValueBinding} instance is called (for example,
when an expression on a JSP tag attribute is being evaluated during the
\textit{render}ing of the page), and the expression is evaluated, and
the result of that evaluation is returned, evaluation takes as follows:

\begin{itemize}
\item The expression language unifies the treatment of the
\emph{.} and \emph{[]} operators.
\emph{expr-a.expr-b} is equivalent to
\emph{a[expr-b]}; that
is, the expression \emph{expr-b} is used to
construct a literal whose value is the identifier, and then the
\emph{[]} operator is used with that value. 
\item The left-most identifier in an expression is evaluated by the
\emph{VariableResolver} instance that is acquired
from the Application instance for this web application. If the value on
the left side of the \emph{.} or
\emph{[]} operator is a
\emph{RowSet}, the object on the right side is
treated as a column name. See the next section for a more complete
evaluation description of these operators. 
\item Each occurrence of the \emph{.} or
\emph{[...]} operators in an expression is evaluated
by the \emph{PropertyResolver }instance that is
acquired from the \emph{Application} instance for
this web application. 
\item Properties of variables are accessed by using the
\emph{.} operator and can be nested arbitrarily. 
\end{itemize}

\section{\bfseries Set Value Semantics}

When the \emph{setValue} method of a
\emph{ValueBinding} is called (for example, for
\emph{text} property bindings for input fields
during Update Model Values), the syntax of the value binding
restriction is restricted as described in the previous section. The
implementation must perform the following processing to evaluate an
expression of the form \emph{\#\{expra.value-b\}} or
\emph{\#\{expr-a[value-b]\}}:

\begin{itemize}
\item Evaluate \emph{expr-a} into
\emph{value-a}. 
\item If \emph{value-a} is null, throw
\emph{PropertyNotFoundException}. 
\item If \emph{value-b} is null, throw
\emph{PropertyNotFoundException}. 
\item If \emph{value-a} is a Map, call
\emph{value-a.put(value-b, new-value)}. 
\item If \emph{value-a} is a
\emph{List} or an array: 

\begin{itemize}
\item Coerce \emph{value-b} to
\emph{int}, throwing
\emph{ReferenceSyntaxException} on an error. 
\item Attempt to execute \emph{value-a.set(value-b,
new-value)} or \emph{Array.set(value-b, new-value)}
as appropriate. 
\item If \emph{IndexOutOfBoundsException }or
\emph{ArrayIndexOutOfBoundsException} is thrown,
throw \emph{PropertyNotFoundException}. 
\item If a different exception was thrown, throw
\emph{EvaluationException}. \newline
\end{itemize}
\item Otherwise (\emph{value-a} is a JavaBeans
object): 

\begin{itemize}
\item Coerce \emph{value-b} to
\emph{String}. 
\item If \emph{value-b} is a writeable property of
\emph{value-a} (as per the JavaBeans Specification),
call the setter method (passing \emph{new-value}).
Throw \emph{ReferenceSyntaxException }if an
exception is thrown. 
\item Otherwise, throw
\emph{PropertyNotFoundException}. \newline
\end{itemize}
\end{itemize}
If the entire expression consists of a single identifier, the following
rules apply:

\begin{itemize}
\item If the identifier matches the name of one of the implicit objects
described below,\newline
throw \emph{ReferenceSyntaxException}. 
\item Otherwise, if the identifier matches the key of an attribute in
\textit{request} scope,\newline
session scope, or application scope, the corresponding attribute value
will be\newline
replaced by \emph{new-value}. 
\item Otherwise, a new \textit{request} scope attribute will be created,
whose key is the\newline
identifier and whose value is \emph{new-value}.
\newline
\end{itemize}
\section{\bfseries Implicit Objects }

The expression language defines a set of implicit objects: 

\begin{itemize}
\item \emph{\textit{FacesContext}} - The
\textit{FacesContext} instance for the current \textit{request}. 
\item \emph{param} - Maps a \textit{request}
parameter name to a single value. 
\item \emph{paramValues} - Maps a \textit{request}
parameter name to an array of values. 
\item \emph{header} - Maps a \textit{request} header
name to a single value. 
\item \emph{headerValues} - Maps a \textit{request}
header name to an array of values. 
\item \emph{cookie} - Maps a cookie name to a single
cookie. 
\item \emph{initParam} - Maps a context
initialization parameter name to a single value. \newline
\end{itemize}
Objects that allow access to various scoped variables:

\begin{itemize}
\item
\emph{\textit{request}}\emph{Scope}
- Maps \textit{request}{}-scoped variable names to their values. 
\item \emph{sessionScope} - Maps session-scoped
variable names to their values. 
\item \emph{applicationScope} - Maps
application-scoped variable names to their values. 
\end{itemize}
When an expression references one of these objects by name, the
appropriate object is returned. An implicit object takes precedence
over an attribute that has the same name. For example,
\emph{\#\{}\emph{\textit{FacesContext}}\emph{\}}
returns the \emph{\textit{FacesContext}} object,
even if there is an existing
\emph{\textit{FacesContext}} attribute containing
some other value.

\section{\bfseries Literals }

The expression language defines the following literals: 

\begin{itemize}
\item Boolean: \emph{true} and
\emph{false} 
\item Integer: as in Java 
\item Floating point: as in Java 
\item String: with single and double quotes;
\emph{} is escaped as
\emph{{\textbackslash}},
\emph{ }is escaped as
\emph{{\textbackslash}}, and
\emph{{\textbackslash}} is escaped as
\emph{{\textbackslash}{\textbackslash}}. 
\item Null: \emph{null}
\end{itemize}

\section{\bfseries Operators }

In addition to the \emph{.} and
\emph{[]} operators discussed above in
\href{http://developers.sun.com/docs/jscreator/help/jsp-jsfel/jsf_expression_language_intro.html#getvaluesemantics}{Get
Value Semantics} and the section after that one, the expression
language provides the following operators: 

\begin{itemize}
\item Arithmetic: \emph{+},
\emph{{}- }\textit{(binary)},
\emph{*}, \emph{/},
\emph{div}, \emph{\%},
\emph{mod}, \emph{{}-}
\textit{(unary)} \newline
\item Logical: \emph{and},
\emph{\&\&}, \emph{or},
\emph{{\textbar}{\textbar}},
\emph{not}, \emph{!} \newline
\item Relational: \emph{==},
\emph{eq}, \emph{!=},
\emph{ne}, \emph{{\textless}},
\emph{lt}, \emph{{\textgreater}},
\emph{gt}, \emph{{\textless}=},
\emph{ge},
\emph{{\textgreater}=},
\emph{le}. Comparisons can be made against other
values, or against boolean, string, integer, or floating point
literals. \newline
\item Empty: The \emph{empty} operator is a prefix
operation that can be used to determine whether a value is
\emph{null} or empty. \newline
\item Conditional: \emph{A ? B : C}. Evaluate
\emph{B} or \emph{C}, depending
on the result of the evaluation of \emph{A}. 
\end{itemize}
The precedence of operators highest to lowest, left to right is as
follows: 

\begin{itemize}
\item \emph{[] .} 
\item \emph{()}~~\textit{(changes precedence of
operators)} 

\begin{itemize}
\item \begin{itemize}
\item \textit{(unary)}\emph{ not ! empty} 
\end{itemize}
\end{itemize}
\item \emph{/ div \% mod} 
\item \emph{+ - }\textit{(binary)} 
\item \emph{{\textless} {\textgreater} {\textless}=
{\textgreater}= lt gt le ge} 
\item \emph{== != eq ne} 
\item \emph{\&\& and} 
\item \emph{{\textbar}{\textbar} or} 
\item \emph{? :}
\end{itemize}

\section{\bfseries Reserved Words }

The following words are reserved for the expression language and must
not be used as identifiers:

\begin{flushleft}
\tablehead{\hline}
\tablelasttail{\hline}
\begin{supertabular}{|c|c|c|c|}
\emph{and} & \emph{false} & \emph{le} & \emph{not} \\
\emph{div} & \emph{ge} & \emph{lt} & \emph{null} \\
\emph{empty} & \emph{gt} & \emph{mod} & \emph{or} \\
\emph{eq} & \emph{instanceof} & \emph{ne} & \emph{true} \\
\end{supertabular}
\end{flushleft}

\bigskip